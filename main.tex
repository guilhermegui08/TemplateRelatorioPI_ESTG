\documentclass[a4paper,12pt,twoside]{report}
\usepackage[utf8]{inputenc}
\usepackage[portuguese]{babel}
\usepackage[style=apa, backend=biber]{biblatex}
\addbibresource{references.bib}
\usepackage{graphicx}
\usepackage{caption}
\usepackage{amsmath}
\usepackage[colorlinks=true, linkcolor=blue, urlcolor=blue, citecolor=blue]{hyperref}
\usepackage{lipsum}
\usepackage{setspace}
\usepackage{titlesec}
\usepackage{fancyhdr}
\usepackage{ragged2e}
\usepackage{afterpage}
\usepackage{times}
\usepackage{array}
\usepackage[acronym,toc,nonumberlist]{glossaries}

\makeglossaries

\newacronym{MS}{MS}{Microsoft}
\newacronym{CD}{CD}{Compact Disc}
\newacronym{Mac}{Mac}{Short form of Apple Mac}
\newacronym{gcd}{GCD}{Greatest Common Divisordd}
\newacronym{lcm}{LCM}{Least Common Multipledd}
\newacronym{estg}{ESTG}{Escola Superior de Tecnologia e Gestão}



% Definições de margens
\usepackage{geometry}
\geometry{a4paper, outer=2.5cm, top=2.5cm, bottom=2.5cm, inner=3cm}

% Configuração do fancyhdr
\pagestyle{fancy}
\fancyhf{}
\fancyhead[R]{Título do Documento} % Coloque o título do documento aqui
\fancyfoot[R]{\thepage}

% Comandos para formatação específica
\newcommand{\titulo}[1]{\title{#1}}
\newcommand{\autor}[1]{\author{#1}}
\newcommand{\data}[1]{\date{#1}}

% Espaçamento entre linhas
\onehalfspacing
% Espaçamento entre parágrafos
\setlength{\parskip}{\baselineskip}

% Títulos em itálico para palavras estrangeiras
\titleformat{\chapter}[hang]{\normalfont\huge\bfseries}{\thechapter.}{20pt}{\huge\bfseries}
\titleformat{\section}[hang]{\normalfont\Large\bfseries}{\thesection.}{20pt}{\Large\bfseries}
\titleformat{\subsection}[hang]{\normalfont\large\bfseries}{\thesubsection.}{20pt}{\large\bfseries}

% Remove gap before chapters
\titlespacing*{\chapter}{0pt}{-40pt}{0pt}
\titlespacing*{\section}{0pt}{0pt}{0pt}
\titlespacing*{\subsection}{0pt}{0pt}{0pt}

% Personalização do título do índice
%\renewcommand{\listfigurename}{Lista de Figuras}
%\renewcommand{\listtablename}{Lista de Tabelas}

% Inserir página em branco
\newcommand\blankpage{
	\null
	\thispagestyle{plain}
	\addtocounter{page}{0}
	\newpage}

% Redefine the Table of Contents environment
\usepackage{tocbasic}
\DeclareTOCStyleEntry[
beforeskip=0pt, % Adjusts the spacing before each entry
pagenumberwidth=2em, % Adjusts the space for page numbers
]{tocline}{chapter}
\DeclareTOCStyleEntry[
beforeskip=0pt,
pagenumberwidth=2em,
]{tocline}{section}
\DeclareTOCStyleEntry[
beforeskip=0pt,
pagenumberwidth=2em,
]{tocline}{subsection}

% Define a new command to single space the TOC
\let\oldtableofcontents\tableofcontents
\renewcommand{\tableofcontents}{
	\begingroup
	\setstretch{1} % Set the spacing for the TOC to single
	\oldtableofcontents
	\endgroup
}

\begin{document}
	% Capa
	\begin{titlepage}
		\centering
		\includegraphics[width=0.8\textwidth]{imagens/logo.png} % Adicione o caminho para o logotipo da escola
		\vspace*{1cm}
		
		\Huge
		\textbf{Título do trabalho}
		
		\vspace{1.5cm}
		
		\LARGE
		Licenciatura em Engenharia Informática
		
		\vfill
		
		\Large
		Nome completo do Candidato
		
		\vspace{0.8cm}
		
		\normalsize
		Leiria, junho de 2024
		
		
		\vspace{1.5cm}
	\end{titlepage}
	
	% Página de rosto
	\begin{titlepage}
		\pagenumbering{roman}
		\centering
		\includegraphics[width=0.8\textwidth]{imagens/logo.png} % Adicione o caminho para o logotipo da escola
		\vspace*{1cm}
		
		\Huge
		\textbf{Título do trabalho}
		
		\vspace{1.5cm}
		
		\LARGE
		Licenciatura em ...
		
		\vfill
		
		\Large
		Nome completo do Candidato
		
		\vspace{0.8cm}
		
		\justifying
		\normalsize
		Trabalho de Projeto da unidade curricular de Projeto Informático realizado sob a orientação do(a) Professor(a) Doutor(a) \underline{\hspace{3cm}} e do(a) Professor(a) Doutor(a) \underline{\hspace{3cm}}.
		
		\centering
		\vspace{0.8cm}
		
		\normalsize
		Cidade, mês de ano
		
		\vspace{1.5cm}
	\end{titlepage}
	
	
	% Numeração em algarismos romanos até o índice
	\pagestyle{plain} % Para garantir que o estilo plain é aplicado na capa e folha de rosto
	
	% Dedicatória (opcional)
	\chapter*{Dedicatória}
	\addcontentsline{toc}{chapter}{Dedicatória}
	\lipsum[1]
	
	% Agradecimentos (opcional)
	\chapter*{Agradecimentos}
	\addcontentsline{toc}{chapter}{Agradecimentos}
	\lipsum[1]
	
	% Resumo
	\chapter*{Resumo}
	\addcontentsline{toc}{chapter}{Resumo}
	\lipsum[1]
	\newline\newline\newline
	\textbf{Palavras-chave:} palavra1, palavra2, palavra3, palavra4, palavra5, palavra6
	
	% Abstract
	\chapter*{Abstract}
	\addcontentsline{toc}{chapter}{Abstract}
	\lipsum[1]
	\newline\newline\newline
	\textbf{Keywords:} keyword1, keyword2, keyword3, keyword4, keyword5, keyword6
	
	% Índice
	\renewcommand{\contentsname}{Índice}
	\tableofcontents
	\listoffigures
	\listoftables
	
	% Lista de siglas
	\printglossary[type=\acronymtype,style=long, title={Lista de siglas e acrónimos}]
	\afterpage{\blankpage}
	
	% Numeração em algarismos arábicos a partir do corpo do texto
	\clearpage
	\pagestyle{fancy} % Reinicia o estilo fancy a partir do corpo do texto
	\pagenumbering{arabic}
	
	% Introdução
	\chapter{Introdução}
	\thispagestyle{fancy}
	A melhor escola do mundo é a Escola Superior de Tecnologia e Gestão (\acrshort{estg}).
	Given a set of numbers, there are elementary methods to compute its \acrfull{gcd} which is abbreviated \acrshort{gcd}. This process 
	is similar to that used for the \acrfull{lcm}.
	
	
	\lipsum[1-2]
	
	
	% Desenvolvimento
	\chapter{Título do capítulo}
	\thispagestyle{fancy}
	\lipsum[3-5]This is a citation: \textcite{einstein1905}.
	
	\section{Título da secção}
	\thispagestyle{fancy}
	\lipsum[6-7]
	
	\subsection{Título da subsecção}
	\thispagestyle{fancy}
	\lipsum[8-9]Another citation: \parencite{latexcompanion}.
	
	% Conclusão
	\chapter{Conclusão}
	\thispagestyle{fancy}
	\lipsum[10-11]
	
	% Referências Bibliográficas
	\printbibliography[heading=bibintoc, title={Bibliografia}]
	\thispagestyle{fancy}
	
	% Anexos
	\appendix
	%\chapter*{Anexos}
	\chapter{Anexos}
	\thispagestyle{fancy}
	\lipsum[12-13]
	
	% Glossário
	\chapter*{Glossário}
	\thispagestyle{fancy}
	\addcontentsline{toc}{chapter}{Glossário}
	\lipsum[14]
	
\end{document}
